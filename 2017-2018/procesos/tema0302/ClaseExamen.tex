\documentclass[a4paper, 12pt]{examen}

\begin{document}

%\modulo{Prog. multim. y de dispositivos moviles}
%\modulo{Planif. y adm. de redes}
\modulo{Prog. de servicios y procesos}
%\modulo{Lenguajes de marcas}




\pregunta{Una empresa necesita una infraestructura basada en servicios que permita transformar la informaci�n de una manera simple por lo que ha pensado en desarrollar una aplicaci�n que permita transformar l�neas de la manera siguiente:

\begin{itemize}
\item{En primer lugar habr� unos emisores de informaci�n que enviar�n l�neas como por ejemplo ``Hola Mundo 1234''.}

\item{En segundo lugar debe haber unos intermediarios que reciban l�neas, las transformen y las reenvien a otros sitios. Por el momento habr� intermediarios de dos tipos. Un tipo de intermediarios tomar�n las l�neas y reemplazar�n las may�sculas por min�sculas y al rev�s.As�, si un intermediario recibe ``Hola Mundo 1234'' deber� generar ``hOLA mUNDO 1234''. El segundo tipo de intermediario tomar� l�neas y reemplazar� todos los n�meros por un signo +. As�, un intermediario que reciba ``Hola Mundo 1234'' deber� generar ``Hola Mundo ++++''.}

\item{En tercer lugar los receptores de informaci�n se limitar�n a recibir l�neas y mostrarlas en pantalla.�}
\end{itemize}
Crear un programa tenga un emisor que env�e 5 l�neas, las reenv�e a un intermediario del tipo 1. Hacer despues que el intermediario del tipo 1 reenv�e su resultado a un intermediario de tipo 2 y que este ultimo lo reenv�e a un receptor. As�, si el emisor env�a ``Hola Mundo 1234'', el resultado final que imprima en pantalla el receptor debe ser ``hOLA mUNDO ++++''.}{10}
\end{document}
