\documentclass{examen}

\begin{document}
\modulo{Programacion de servicios y procesos}

En los siguientes ejercicios vas a tener que trabajar con claves p�blicas. Recuerda lo siguiente:

\begin{itemize}
\item{Vas a tener que enviar varias cosas al profesor: los ejercicios del examen, tu almac�n de claves, la clave de dicho almac�n con tu certificado y los ficheros Java asociados.}
\item{Si algo no te va bien y decides borrar el almac�n de claves {\bf vas a tener que volver a generar tu certificado}}
\end{itemize}
\pregunta{Fabrica un almacen de claves con la contrase�a (sin comillas) ``ABCDABCD'' y pon en �l tu informaci�n personal. A partir de �l extraer un certificado con el nombre de fichero Certficado-tu nombre y apellido.cer Construye un JAR que imprima tu nombre y DNI y f�rmalo con dicho certificado. Modif�ca el JAR metiendo dentro una clase que imprima el mensaje ``Hola mundo'' y comprueba que ahora la verificaci�n de seguridad falla.


Se vuelve a recordar que en este ejercicio tienes que entregar:
\begin{itemize}
\item{El fichero de almac�n.}
\item{El fichero de certificado}
\item{El primer JAR (el que imprime tu DNI) con su c�digo fuente.}
\item{El segundo JAR (el que imprime "Hola mundo) con su c�digo fuente}
\end{itemize}
}
{10}

\end{document}
