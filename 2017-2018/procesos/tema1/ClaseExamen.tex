\documentclass[a4paper, 12pt]{examen}
\usepackage{inputenc}
\begin{document}

%\modulo{Prog. multim. y de dispositivos moviles}
%\modulo{Planif. y adm. de redes}
\modulo{Prog. de servicios y procesos}
%\modulo{Lenguajes de marcas}


\pregunta{ Un programador necesita una herramienta que pueda buscar palabras
en ficheros. Sin embargo, el rendimiento de las herramientas que tiene es muy bajo, por
lo que desea crear la suya propia con capacidades multiproceso. Para ello, se ha marcado los siguientes requisitos:  
\begin{itemize}
\item{La herramienta recibe diversos par�metros: primero recibe la palabra a buscar y despues puede recibir muchos nombres de fichero en los que debe buscar. Se asume que siempre nos pasar�n como m�nimo un nombre de fichero por lo que {\em NO HACE FALTA COMPROBAR ERRORES RELATIVOS A QUE HAYA 0 NOMBRES DE FICHERO}.}
\item{El programa buscar� en paralelo la palabra pasada en todos los ficheros que se le pasen como par�metro. Para ello habr� un proceso por fichero.}
\item{La aplicaci�n siempre generar� un fichero llamado ``Resultados.txt'' en los que aparecer�n muchas l�neas. Cada l�nea indicar� tres cosas, el nombre del fichero, la palabra buscada y el n�mero de veces que aparece. Evidentemente, la palabra buscada siempre ser� la misma, pero es necesario que la palabra aparezca.}
\item{La aplicaci�n busca coincidencias exactas, por lo que se considera que {\tt palabra!=Palabra}}

\end{itemize}
}{ 10 }

\break
Al terminar el proyecto, se deber� generar un ZIP (no RAR) con todo el c�digo fuente de la aplicaci�n y entreg�rselo al profesor.

Para facilitar las pruebas, se adjuntan cuatro ficheros con los nombres siguientes y con algunos resultados de prueba:
\begin{itemize}
\item{Fichero {\tt fichero1.txt}: contiene la palabra ``Java'' 2 veces y la palabra ``lenguaje'' 1 sola vez.}
\item{Fichero {\tt fichero2.txt}: contiene la palabra ``Java'' 1 sola vez y la palabra ``lenguaje'' 1 sola vez.}
\item{Fichero {\tt fichero3.txt}: no contiene ``Java'' ni ``lenguaje''.}
\item{Fichero {\tt fichero4.txt}: no contiene ``Java'' pero s� ``lenguaje'' 1 sola vez.}

\end{itemize}

El programador desea simplificarse al m�ximo la ejecuci�n por lo que {\em TODO EL C�DIGO DE LA APLICACI�N IR� EN UN SOLO PROYECTO}. Si necesita alguna clase adicional la puede copiar y pegar a un fichero nuevo del proyecto.



\end{document}
