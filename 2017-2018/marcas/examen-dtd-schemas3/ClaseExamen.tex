\documentclass{examen}

\begin{document}
\modulo{Lenguajes de marcas}

\pregunta{Elaborar un esquema XML que permita validar un fichero XML que se atenga a las reglas siguientes:}{6.5}
\begin{itemize}

\item{El elemento ra�z se llama {\tt productos}. Dentro de �l debe haber 1 o m�s elementos {\tt monitor} o {\tt cpu}}
\begin{itemize}
\item{Un {\tt monitor} lleva dentro 3 cosas: un elemento {\tt resoluci�n} (obligatorio y contiene cadenas), un elemento {\tt peso} (obligatorio, contiene positivos, y un elemento {\tt enstock} (optativo, siempre est� vac�o)}.
\item{La {\tt resoluci�n} lleva siempre dos atributos obligatorios {\tt ancho} y {\tt alto} que contienen positivos. Dentro de la etiqueta {\tt resoluci�n} siempre hay una cadena de texto}
\end{itemize}
\item{ Una {\tt CPU } lleva dos cosas dentro: }
\begin{itemize}
\item {Un elemento {\tt tarjetagrafica} que debe llevar dentro siempre uno de estos 3 valores: Nvidia, 3dfx o AMD}
\item{Un elemento {\tt procesador } que lleva dentro una etiqueta. PUEDE llevar un atributo {\tt codigo} que siempre tiene una estructura definida por 3 may�sculas, seguidas de un gui�n, seguidas de 3 n�meros.}
\end{itemize}
\end{itemize}

\break

\pregunta{Crear una DTD que permita validar un fichero XML con las mismas reglas indicadas anteriormente}{3.5}

\begin{verbatim}
<!--Ejemplo de fichero para los ejercicios-->
<productos>
    <monitor>
        <resolucion ancho="1920" alto="1600">FullHD</resolucion>
        <peso>20</peso>
        <enstock/>
    </monitor>
    <cpu>
        <grafica>Nvidia</grafica>
        <procesador codigo="ABC-123">Intel</procesador>
    </cpu>
</productos>
\end{verbatim}
\end{document}
