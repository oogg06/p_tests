\documentclass{examen}

\begin{document}
\modulo{Lenguajes de marcas}

\pregunta{Elaborar una DTD que permita validar un fichero XML que se atenga a las reglas siguientes:}{3.5}
\begin{itemize}
\item{    El elemento ra�z se llama listacomponentes.}
\item{    Dentro de �l puede haber uno o m�s elementos componente}
\item{    Un componente puede ser una tarjetagrafica o un monitor.}
\item{    Un componente puede tener un atributo llamado codigo cuya estructura es siempre un d�gito de 6 cifras.}
\item{    Una tarjeta gr�fica siempre tiene dos elementos llamados memoria y precio.}
\item{    La memoria siempre es una cifra seguido de GB o TB.}
\item{    El tama�o del monitor siempre es un entero positivo.}
\item{    El precio siempre es una cantidad positiva con decimales. El precio siempre lleva un atributo moneda que solo puede valer ``euros'' o ``dolares'' y que se utiliza para saber en qu� moneda est� el precio.}

\end{itemize}

\break

\pregunta{Crear un fichero de esquema XML que permita validar un fichero XML como el mostrado al final y para el cual se han definido las siguientes reglas}{6.5}
\begin{itemize}
\item{    El elemento ra�z se llama listacomponentes.}
\item{    Dentro de �l puede haber uno o m�s elementos componente}
\item{    Un componente puede ser una tarjetagrafica o un monitor.}
\item{    Un componente puede tener un atributo llamado codigo cuya estructura es siempre un d�gito de 6 cifras.}
\item{    Una tarjeta gr�fica siempre tiene dos elementos llamados memoria y precio.}
\item{    La memoria siempre es una cifra seguido de GB o TB.}
\item{    El tama�o del monitor siempre es un entero positivo.}
\item{    El precio siempre es una cantidad positiva con decimales. El precio siempre lleva un atributo moneda que solo puede valer ``euros'' o ``dolares'' y que se utiliza para saber en qu� moneda est� el precio.}

\end{itemize}


\begin{verbatim}
<listacomponentes>
  <componente>
    <tarjetagrafica>
      <memoria>2GB</memoria>
      <precio moneda="euros">190</precio>
    </tarjetagrafica>
  </componente>
  <componente codigo="123456">
    <monitor>
      <tamanio>14</tamanio>
      <precio moneda="euros">99.49</precio>
    </monitor>
  </componente>
</listacomponentes>
\end{verbatim}
\end{document}
