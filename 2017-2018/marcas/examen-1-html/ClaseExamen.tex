\documentclass[a4paper, 12pt]{examen}

\begin{document}

%\modulo{Prog. multim. y de dispositivos moviles}
%\modulo{Planif. y adm. de redes}
%\modulo{Prog. de servicios y procesos}
\modulo{Lenguajes de marcas}


\pregunta{ Elabora un fichero HTML que consiga exactamente
lo que se muestra en la figura \ref{figura2}. En este ejercicio se debe escribir todo el HTML, incluyendo cabecera,
cuerpo y elementos relevante para la estructura.}{  3.5 }
\begin{figure}[h]
    \caption{Resultado esperable en el ejercicio 1}
    \label{figura2}
    \includegraphics[width=\linewidth]{ej2.png}
\end{figure}
\break

\pregunta{ Elabora un fichero HTML que consiga exactamente
lo que se muestra en la figura \ref{figura1}. En este ejercicio solo hace falta escribir el HTML relevante,
en este caso solo a partir de la etiqueta {\tt table}}{  3 }
\begin{figure}[h]
    \caption{Resultado esperable en el ejercicio 1}
    \label{figura1}
    \includegraphics[width=\linewidth]{ej1.png}
\end{figure}
\break

\pregunta{ Elabora un fichero HTML que consiga exactamente
lo que se muestra en la figura \ref{figura3}. En este ejercicio solo hace falta escribir el HTML relevante,
en este caso solo a partir de la etiqueta {\tt form} }{  3.5 }
\begin{figure}[h]
    \caption{Resultado esperable en el ejercicio 3}
    \label{figura3}
    \includegraphics[width=\linewidth]{ej3.png}
\end{figure}


\end{document}
