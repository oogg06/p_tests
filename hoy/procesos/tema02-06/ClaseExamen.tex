\documentclass[a4paper, 12pt]{examen}
\usepackage{inputenc}
\begin{document}

%\modulo{Prog. multim. y de dispositivos moviles}
%\modulo{Planif. y adm. de redes}
\modulo{Prog. de servicios y procesos. Recuperaci�n}
%\modulo{Lenguajes de marcas}
\pregunta{Se desea simular el comportamiento de las colas de clientes en una gasolinera en base a los siguientes requisitos:

\begin{itemize}
\item{Hay tres surtidores con 100 litros cada uno.}
\item{Los clientes no llegan todos a la vez. Entre la generaci�n de un cliente y otro hay un tiempo aleatorio de entre 1 y 3 segundos. Se generar�n como m�ximo 10 clientes.}

\item{Cuando se genera un cliente �ste intenta acceder a algun surtidor.Si hay alguno libre lo ocupa durante un tiempo al azar entre 1 y 3 segundos y extrae entre 20 y 40 litros de gasolina. Puede pasar que est�n todos los surtidores ocupados, en ese caso el cliente se marcha. Tambi�n puede pasar que un surtidor est� libre pero no tenga gasolina. Un cliente no coge un surtidor si no queda suficiente gasolina para �l.}
\item{Se requiere almacenar cuanta gasolina coge cada cliente. Debe ocurrir que la suma total de gasolina de todos los cliente coincida con la cantidad de gasolina que falta en los surtidores. Es decir, si hay tres surtidores con 100 litros (que suman 300 litros) y despues de la ejecuci�n ocurre que entre los tres surtidores hay 100 litros (lo que significa que supuestamente se han extra�do 200 litros) y entre todos los clientes suman 300 litros de gasolina extra�da se podr� deducir con total seguridad que hay un error.}
\end{itemize}
}{10}

\end{document}
