\documentclass[a4paper, 12pt]{examen}

\begin{document}

%\modulo{Prog. multim. y de dispositivos moviles}
%\modulo{Planif. y adm. de redes}
\modulo{Prog. de servicios y procesos}
%\modulo{Lenguajes de marcas}




Se desea crear un servicio de transporte de datos que sea capaz de modificar las cadenas que recibe convirtiendo a may�sculas todos las cadenas que reciba. En este sistema habr� emisores de mensajes, receptores de mensajes y servidores intermediarios. 

Un emisor de mensajes se dedicar� a conectarse a un servidor y enviar de 1 a 10 mensajes (cantidad al azar) de uno de estos mensajes:

\begin{itemize}
\item{\tt Mensaje}
\item{\tt Texto de prueba}
\item{\tt Canal abierto}
\end{itemize}

El servidor de mensajes se dedicar� a recibir mensajes de los emisores, transformarlos como se ha dicho y reenviarlos a los receptores.

El receptor de mensajes se dedicar� a estar en un bucle infinito recibiendo mensajes del servidor en imprimi�ndolos en pantalla.

\pregunta{Crear el emisor de mensajes en Java}{2}
\pregunta{Crear el receptor de mensajes}{2}
\pregunta{Crear un servidor multihilo que se comporte de acuerdo a lo especificado en el texto anterior.}{6}


\end{document}
