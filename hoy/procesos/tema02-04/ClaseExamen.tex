\documentclass[a4paper, 12pt]{examen}
\usepackage{inputenc}
\begin{document}

%\modulo{Prog. multim. y de dispositivos moviles}
%\modulo{Planif. y adm. de redes}
\modulo{Prog. de servicios y procesos}
%\modulo{Lenguajes de marcas}
\pregunta{Se desea programar un sistema que permita registrar la actividad de los distintos procesos que est�n en ejecuci�n. Existen N ficheros llamados ``fich1.txt'', ``fich2.txt'', ``fichN.txt'' en el cual se van a registrar eventos en forma de mensajes. Puede haber M hilos de ejecuci�n Java que se dedican a escribir mensajes (entre 1 y 10 mensajes) con un espacio de tiempo de entre 1 y 3 segundos entre cada mensaje. Crear un programa Java que realice esta simulaci�n correctamente, evitando el bloqueo, la inanici�n y por supuesto maximizando la concurrencia. }{10}

En cuanto a la simulaci�n se debe tener en cuenta lo siguiente:

\begin{itemize}


\item{Se permite el uso de funciones de biblioteca que ya se tengan en el ordenador (como por ejemplo la biblioteca Utilidades creada en clase que permite elegir n�meros al azar o esperar tiempos al azar) pero no se permite el uso de Internet.}

\item{Para acelerar las pruebas puede ser recomendable usar al principio menos procesos, y con menos tiempo de espera entre intentos. Para facilitar las cosas puede ser �til el usar constantes que nos permitan cambiar estas cantidades con facilidad.}

\item{Al finalizar se debe hacer un ZIP con las clases Java y enviarlas por correo al profesor.}
\end{itemize}
Escribir el programa Java que haga la simulaci�n de manera correcta.

\end{document}
