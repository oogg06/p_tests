\documentclass[a4paper, 12pt]{examen}
\usepackage{inputenc}
\begin{document}

%\modulo{Prog. multim. y de dispositivos moviles}
%\modulo{Planif. y adm. de redes}
\modulo{Prog. de servicios y procesos}
%\modulo{Lenguajes de marcas}


\pregunta{ Un programador necesita una herramienta que pueda buscar palabras
en ficheros. Sin embargo, el rendimiento de las herramientas que tiene es muy bajo, por
lo que desea crear la suya propia con capacidades multiproceso. Para ello, se ha marcado los siguientes requisitos: }{10}
\begin{itemize}
\item{La herramienta recibe diversos par�metros: primero recibe un par�metro que puede ser {\tt ascendente} o {\tt descendente} y despues puede recibir muchos nombres de fichero. Se asume que siempre nos pasar�n como m�nimo un nombre de fichero por lo que {\em NO HACE FALTA COMPROBAR ERRORES RELATIVOS A QUE HAYA 0 NOMBRES DE FICHERO}.}
\item{El programa crear� tantos procesos hijo como ficheros tenga que procesar. Un hijo recibe el par�metro {\tt ascendente} o {\tt descendente} y un solo nombre  de fichero y ordena las l�neas de ese fichero. Las l�neas se almacenar�n en un fichero cuyo nombre el mismo nombre que se le pas� pero con la palabra {\tt Resultado} delante.}


\end{itemize}


\break
Al terminar el proyecto, se deber� generar un ZIP (no RAR) con todo el c�digo fuente de la aplicaci�n y entreg�rselo al profesor.

Para facilitar las pruebas, se adjuntan cuatro ficheros con los nombres {\tt fichero1.txt}, {\tt fichero2.txt}, {\tt fichero3.txt} y {\tt fichero4.txt}


El programador desea simplificarse al m�ximo la ejecuci�n por lo que {\em TODO EL C�DIGO DE LA APLICACI�N IR� EN UN SOLO PROYECTO}. Si necesita alguna clase adicional la puede copiar y pegar a un fichero nuevo del proyecto.



\end{document}
