\documentclass[a4paper, 12pt]{examen}
\usepackage{inputenc}
\begin{document}

%\modulo{Prog. multim. y de dispositivos moviles}
%\modulo{Planif. y adm. de redes}
\modulo{Prog. de servicios y procesos}
%\modulo{Lenguajes de marcas}
\pregunta{Se desea programar un gestor de memoria que sea capaz de almacenar n�meros enteros y que pueda ser utilizado por procesos que trabajen con concurrencia. Para ello se han definido los siguientes requisitos:
\begin{itemize}
\item{El gestor de memoria puede almacenar hasta 5 n�meros enteros. Tambi�n debe poder permitir leer un n�mero almacenado en una posici�n, y liberar memoria as� que debe tener m�todos {\tt almacenar}, {\tt liberar} y {\tt leer.}}

\item{Cuando un proceso quiere almacenar algo puede que haya una posici�n libre o puede que no. El gestor de memoria devolver� un {\tt int} que puede ser una posici�n de 0 a 4 indicando que se ha podido almacenar algo o un -1 para indicar que no quedaba memoria libre.}

\item {El gestor de memoria debe llevar el control de qu� posiciones est�n ocupadas y cuales no. }

\item{Cuando un proceso quiere leer algo del gestor de memoria debe pasar la posici�n de la que desea leer. Evidentemente, una posici�n de memoria que est� libre podr� ser reusada por otro proceso.}
\item{Cuando un proceso quiere liberar memoria debe pasar la posici�n que desea liberar.}

\end{itemize}


}{10}

En cuanto a la simulaci�n se debe tener en cuenta lo siguiente:

\begin{itemize}

\item {Habr� 1000 procesos que usar�n el mismo gestor de memoria. Cada proceso har� 100 intentos de escribir un n�mero al azar en el gestor de memoria. }

\item {Si tras esos 100 intentos no lo consigue termina. }

\item {Si en alg�n momento consigue almacenar su n�mero espera un tiempo al azar entre 100 y 500 milisegundos comprueba si en el gestor est� el mismo n�mero que hab�a almacenado, libera esa posici�n de memoria y termina.}

\item{Se permite el uso de funciones de biblioteca que ya se tengan en el ordenador (como por ejemplo la biblioteca Utilidades creada en clase que permite elegir n�meros al azar o esperar tiempos al azar) pero no se permite el uso de Internet}

\item{Para acelerar las pruebas puede ser recomendable usar al principio menos procesos, con menos intentos y con menos tiempo de espera entre intentos. Para facilitar las cosas puede ser �til el usar constantes que nos permitan cambiar estas cantidades con facilidad.}
\end{itemize}
Escribir el programa Java que haga la simulaci�n de manera correcta.

\end{document}
