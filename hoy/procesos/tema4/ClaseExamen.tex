\documentclass{examen}

\begin{document}
\modulo{Programacion de servicios y procesos}

En los siguientes ejercicios vas a tener que trabajar con claves p�blicas. Recuerda lo siguiente:

\begin{itemize}
\item{Vas a tener que enviar varias cosas al profesor: los ejercicios del examen, tu almac�n de claves, la clave de dicho almac�n y tu certificado.}
\item{Si algo no te va bien y decides borrar el almac�n de claves {\bf vas a tener que volver a generar tu certificado}}
\end{itemize}
\pregunta{Fabrica una pareja de claves indicando tu informaci�n personal (nombre completo, nombre del instituto, JCCM, Ciudad Real, Ciudad Real, ES) y exporta tu certificado. En el nombre de fichero debe aparecer tu nombre y apellidos (por ejemplo {\tt Certificado-Jose-Sanchez.cer}).
Crea un programa que act�e como servidor multihilo que trabaje de la manera siguiente:
\begin{itemize}
\item {El servidor leer� una l�nea que le enviar� el cliente y que contendr� texto.}
\item {El servidor leer� una l�nea que contiene una sola letra.}
\item {El servidor contar� cuantas vocales hay en la l�nea (A, E, I, O, U, a, e, i, o, u).}
\item {El servidor enviar� al cliente una l�nea con dicho n�mero.}
\item {El servidor enviar� una l�nea con el texto ``Realizado por:'' junto con tu nombre, apellidos y DNI}
\item{En pocas palabras el servidor recibe dos l�neas y luego env�a dos l�neas. No olvides usar {\tt println()} y {\tt flush()}}
\end{itemize}
}{7}
\pregunta{Crea una clase Java llamada Saludo en el que el main imprima tu nombre y apellidos. Fabrica un JAR, f�rmalo y env�aselo al profesor (quiz�s te interese verificar la firma antes de enviarlo). }{3}
\end{document}
