\documentclass[a4paper, 12pt]{examen}

\begin{document}

%\modulo{Prog. multim. y de dispositivos moviles}
%\modulo{Planif. y adm. de redes}
\modulo{Prog. de servicios y procesos}
%\modulo{Lenguajes de marcas}




\pregunta{Un servicio de validaci�n funciona en base al siguiente protocolo:

\begin{enumerate}
\item{El cliente env�a un n�mero X entre 2 y 5}
\item{El cliente espera a recibir del servidor X n�meros que el servidor genera de modo aleatorio}
\item{El cliente debe elegir el m�s peque�o y el m�s grande (podr�an ser iguales). El cliente env�a dos l�neas separadas, enviando en una el n�mero m�s peque�o y en otra el m�s grande.}
\item{El servidor convierte a {\tt float} ambos n�meros y divide el grande por el peque�o.  Env�a el cociente al cliente.}
\item{El cliente tambi�n divide el grande por el peque�o, recibe una l�nea con el cociente que calcul� el servidor y comprueba si son iguales. El cliente escribe en pantalla su cociente y el cociente del servidor. Si el cociente que recibe del servidor y el que �l calcula son iguales env�a OK al servidor. Si no son iguales, env�a FAIL.}
\end{enumerate}

Crea un cliente Java que implemente este protocolo. Se valorar� la calidad del c�digo de la soluci�n.}{3.5}


\pregunta{Implementa el servidor multihilo que procesa el protocolo comentado en el ejercicio anterior. Se valorar� la calidad del c�digo de la soluci�n.}{6.5}


\end{document}
