\documentclass{examen}

\begin{document}
\modulo{Lenguajes de marcas y sistemas de gesti�n de informaci�n}
\pregunta{
Una empresa desea modelar un flujo que indique las novedades en su canal de ventas, por lo que desea ver como quedar�a un archivo RSS que refleje dichas novedades.
\begin{itemize}
\item{La URL principal es http://acme.com y el canal de la empresa se llamar� ``Novedades de ACME'' siendo su descripci�n ``Las m�s recientes novedades al servicio de nuestros clientes''}
\item{Dentro de dicho canal se desea ver noticias de ejemplo}
\begin{enumerate}
	\item{La primera apunta a la URL http://acme.com/novedades1 su descripci�n es ``Disponible la nueva actualizaci�n de Android en los servidores de Google y r�plicas autorizadas`` y el t�tulo ``Nueva versi�n de Android''}
	\item{La segunda noticia tiene la URL http://acme.com/novedades2, su t�tulo es ``Revisi�n de HTML'' y la descripci�n es ``Cambios en el lenguaje de marcas para la web''}
\end{enumerate}

\end{itemize}

}{1}

\break 
\pregunta{Dado el archivo de pedido XML con el nombre ``Fichero original'' y que contiene informaci�n sobre ordenadores, 
resolver con XQuery estas consultas
\begin{itemize}
\item{Extraer el c�digo de los port�tiles que lleven disco SSD.}
\item{Extraer toda la informaci�nd e los port�tiles con un precio superior a 400.}
\item{Extraer el peso de los port�tiles que lleven disco SSD.}
\end{itemize}
}{3}

\pregunta{Transformar el archivo de pedido XML con el nombre ``Fichero original'' en una p�gina HTML que contenga una tabla con los equipos que contengan 4GB de memoria o m�s.}{6}		

\break

\begin{lstlisting}{language=xml}
<!--FICHERO ORIGINAL-->
<pedido>
	<portatiles>
		<portatil codigo="C1">
			<peso>1430</peso>
			<ram unidad="GB">4</ram>
			<disco tipo="ssd">500</disco>
			<precio>499</precio>
		</portatil>
		<portatil codigo="C2">
			<peso>1830</peso>
			<ram unidad="GB">1</ram>
			<disco tipo="ssd">1000</disco>
			<precio>1199</precio>
		</portatil>
		<portatil codigo="C3">
			<peso>1250</peso>
			<ram unidad="MB">2048</ram>
			<disco tipo="ssd">750</disco>
			<precio>699</precio>
		</portatil>
	</portatiles>
</pedido>

\end{lstlisting}

\break

\begin{lstlisting}{language=xml}
<!--Fichero que debe salir como resultado del ejercicio 3-->
<html>
	<head>
		<title>Examen</title>
	</head>
	<body>
		<table border="1">
			<tr>
				<th>Codigo de dispositivo</th>
				<th>Memoria</th>
			</tr>
			<tr>
				<td>C1</td>
				<td>4GB</td>
			</tr>
		</table>
	</body>
</html>

\end{lstlisting}
\end{document}