\documentclass{examen}

\begin{document}
\modulo{Lenguajes de marcas y sistemas de gesti�n de informaci�n}
\pregunta{
Una empresa desea modelar un flujo que indique las novedades en su canal de ventas, por lo que desea ver como quedar�a un archivo RSS que refleje dichas novedades.
\begin{itemize}
\item{La URL principal es http://acme.com y el canal de la empresa se llamar� ``Novedades de ACME'' siendo su descripci�n ``Las m�s recientes novedades al servicio de nuestros clientes''}
\item{Dentro de dicho canal se desea ver noticias de ejemplo}
\begin{enumerate}
	\item{La primera apunta a la URL http://acme.com/novedades1 su descripci�n es ``Disponible la nueva actualizaci�n de Android en los servidores de Google y r�plicas autorizadas`` y el t�tulo ``Nueva versi�n de Android''}
	\item{La segunda noticia tiene la URL http://acme.com/novedades2, su t�tulo es ``Fin de XP'' y la descripci�n es ``Finaliz� el soporte de Microsoft para Windows XP''}
\end{enumerate}

\end{itemize}

}{1}


\pregunta{Dado el fichero de inventario que se puede encontrar con el comentario ``Fichero de inventario para consultas XPath'' resolver las siguientes consultas:
\begin{itemize}
\item{Extraer el nombre (sin etiquetas) de los productos que lleven el peso en gramos.}
\item{Extraer el peso de los productos (sin etiquetas) cuyo codigo sea ``DEZ-138'' o tengan la unidad del peso en gramos}
\end{itemize}
}{2.5}

\pregunta{Dado el archivo XML con el comentario ``Fichero para transformar'', 
crear una hoja de estilo XSLT que lo transforme el fichero de resultado que se muestra con el comentario ``Fichero que debe salir como resultado del XSLT''. Observar que en el resultado:
hay dos sublistados: uno para grandes clientes, con {\tt ventastotales} mayores de 100.000 sin importar la moneda, y otro sublistado para clientes con menos de 100.000.
}{6.5}



\break 
\begin{lstlisting}{language=xml}
<!--Fichero para transformar-->
<clientes>
    <cliente>
        <ventastotales moneda="euro">134000</ventastotales>
        <nombre pais="US">
            <nombrepila>Bob</nombrepila>
            <apellido1>Smith</apellido1>
        </nombre>
    </cliente>
    <cliente>
        <ventastotales moneda="dolar">29000</ventastotales>
        <nombre pais="US">
            <nombrepila>Alice</nombrepila>
            <apellido1>Rogers</apellido1>
        </nombre>
    </cliente>
    <cliente>
        <ventastotales moneda="euro">51500</ventastotales>
        <nombre pais="ES">
            <nombrepila>Carmen</nombrepila>
            <apellido1>Garcia</apellido1>
        </nombre>
    </cliente>
</clientes>
\end{lstlisting}

\break

\begin{lstlisting}{language=xml}
<!--Fichero que debe salir como resultado del XSLT-->
<listado>
    <grandesclientes>
        <cliente nombre="Bob Smith">
            <nacionalidad>US</nacionalidad>
            <transacciones total="134000" moneda="euro"/>
        </cliente>
    </grandesclientes>
    <pequenosclientes>
        <cliente nombre="Alice Rogers">
            <nacionalidad>US</nacionalidad>
            <transacciones total="29000" moneda="dolar"/>
        </cliente>
        <cliente nombre="Carmen Garcia">
            <nacionalidad>ES</nacionalidad>
            <transacciones total="51500" moneda="euro"/>
        </cliente>
    </pequenosclientes>
</listado>
\end{lstlisting}



\break


\begin{lstlisting}{language=xml}
<!--Fichero de inventario para consultas XPath-->
<inventario>
    <producto codigo="AAA-111">
        <nombre>Teclado</nombre>
        <peso unidad="g">480</peso>
    </producto>
    <producto codigo="ACD-981">
        <nombre>Monitor</nombre>
        <peso unidad="kg">1.8</peso>
    </producto>
    <producto codigo="DEZ-138">
        <nombre>Raton</nombre>
        <peso unidad="g">50</peso>
    </producto>
</inventario>

\end{lstlisting}


\end{document}
