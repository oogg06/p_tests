\documentclass{examen}

\begin{document}
\modulo{Lenguajes de marcas}

\pregunta{Elaborar un esquema XML que permita validar un fichero XML que se atenga a las reglas siguientes:}{6.5}
\begin{itemize}

\item{El elemento raíz se llama {\tt articulos}. Dentro ocurre lo siguiente.}
\item{Hay uno o más elementos {\tt ordenador} con un atributo optativo
    llamado ``lote''. Si aparece tiene la estructura
    {\tt AA-2222} (dos mayúsculas, seguidas de guión, seguidas de cuatro cifras}
\item{Dentro de la etiqueta {\tt procesador} hay cadenas, pero hay un atributo
        obligatorio llamado ``marca'' que siempre contiene ``AMD'' o ``Intel''}
\item{Dentro de la etiqueta ordenador PUEDE haber un elemento llamado
        ``pulgadasmonitor'' que siempre lleva  dentro un entero que puede
        valer de 13 a 42}        
\end{itemize}

\break

\pregunta{Crear una DTD que permita validar un fichero XML con las mismas reglas indicadas anteriormente o al menos todas las que pueda controlar una DTD}{3.5}

\begin{verbatim}
<!--Ejemplo de fichero para los ejercicios-->
<!--El elemento raíz se llama artículos-->
<articulos>
    <!--Hay uno o más elementos ordenador con un atributo optativo
    llamado ``lote'' que es optativo. Si aparece tiene la estructura
    AA-2222 (dos mayúsculas, seguidas de guión, seguidas de cuatro cifras)-->
    <ordenador lote="FZ-5192">
        <!--Dentro de la etiqueta procesador hay cadenas, pero hay un atributo
        obligatorio llamado ``marca'' que siempre contiene ``AMD'' o ``Intel''-->
        <procesador marca="AMD">Equivalente i7</procesador>
        <!--Dentro de la etiqueta ordenador PUEDE haber un elemento llamado
        ``pulgadasmonitor'' que siempre lleva un dentro un entero que puede
        valer de 13 a 42-->
        <pulgadasmonitor>15</pulgadasmonitor>
    </ordenador>
    <ordenador>
        <procesador marca="Intel">Pentium G4250</procesador>
    </ordenador>
</articulos>
\end{verbatim}
\end{document}
