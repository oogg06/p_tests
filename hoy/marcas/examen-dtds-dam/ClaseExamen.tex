\documentclass{examen}

\begin{document}
\modulo{Lenguajes de marcas}

\pregunta{Elaborar un XML Schema que permita validar un fichero XML como el mostrado al final y que se atenga a las reglas siguientes:}{6.5}
\begin{itemize}
\item{    El elemento raíz se llama ``listaarticulos''}
\item{    Dentro de él puede haber uno o más elementos ``articulo''. Todo artículo tiene siempre un atributo código formado por 3 o 4 cifras seguidas de 4 letras mayusculas.}
\item{  Dentro de un artículo hay estos elementos: ``peso'' (que es optativo), , ``descripcion'', ``fechaentrega'' y un cuarto elemento ``pagado'' que contiene dentro una cadena indicando la situación del pago.}
\item{   El elemento ``peso'' puede o no tener un atributo llamado ``unidad''. Dicha unidad siempre es ``g'', ``kg'' o ``ton''. Dentro del peso hay un número que puede tener decimales y que siempre es mayor que 0.}
\item{La descripción siempre es un texto.}
\item{La fecha siempre tiene la estructura AAAA-MM-DD, siendo AAAA el año (siempre con cuatro cifras), MM el mes (siempre dos cifras, por ejemplo 03 o 10) y DD el dia (siempre dos cifras, por ejemplo 01 o 13).}

\end{itemize}

\break

\pregunta{Crear una DTD que permita validar un fichero XML como el mostrado al final y para el cual se han definido las mismas reglas que para el ejercicio anterior.}{3.5}



\begin{verbatim}
<listaarticulos>
    <articulo codigo="123ABCD">
        <peso unidad="kg">300</peso>
        <descripcion>Articulo AAA</descripcion>
        <fechaentrega>2020-02-19</fechaentrega>
        <pagado>Parcialmente</pagado>
    </articulo>
    <articulo codigo="5678AXYZ">
        <!--No tiene peso-->
        <descripcion>Articulo ABC</descripcion>
        <fechaentrega>2021-11-23</fechaentrega>
        <pagado>Si, por completo</pagado>
    </articulo>
</listaarticulos>
\end{verbatim}
\end{document}

