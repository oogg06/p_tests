\documentclass{examen}

\begin{document}
\modulo{Lenguajes de marcas y sistemas de gesti�n de informaci�n}

Dado el archivo XML que se puede encontrar al final, extraer la informaci�n pedida en los siguientes enunciados usando el lenguaje que se indique

\pregunta{Recuperar el nombre de proveedor (con la etiqueta incluida) de los proveedores cuyo estado sea 20}{1}

\pregunta{Recuperar el nombre de proveedor y el nombre de proveedor que est�n en las mismas ciudades}{1.5}
\pregunta{Averiguar la suma total de cantidades suministradas por los proveedores v1 y v4. El resultado es una sola cantidad}{1.5}
\pregunta{Obtener las cantidades suministradas a los proyectos cuya ciudad sea Atenas. El resultado es una sola cantidad.}{2}
\pregunta{Obtener la suma de cantidades donde el proveedor tuviera como ciudad Roma (el resultado es una sola cantidad, la suma total)}{2}
\pregunta{Obtener la media de cantidades suministradas para los proveedores Smith y Jones}{2}


\break 
\begin{verbatim}
<datos>
    <proveedores>
        <proveedor numprov="v1">
            <nombreprov>Smith</nombreprov>
            <estado>20</estado>
            <ciudad>Londres</ciudad>
        </proveedor>
        ... omitido ...
    </proveedores>
    <partes>
        <parte numparte="p1">
            <nombreparte>Tuerca</nombreparte>
            <color>Rojo</color>
            <peso>12</peso>
            <ciudad>Londres</ciudad>
        </parte>
        ... omitido ...
    </partes>
    <proyectos>
        <proyecto numproyecto="y1">
            <nombreproyecto>Clasificador</nombreproyecto>
            <ciudad>Paris</ciudad>
        </proyecto>
        ... omitido ...
    </proyectos>
    <suministros>
        <suministra>
            <numprov>v1</numprov>
            <numparte>p1</numparte>
            <numproyecto>y1</numproyecto>
            <cantidad>200</cantidad>
        </suministra>
        <suministra>
            <numprov>v1</numprov>
            <numparte>p1</numparte>
            <numproyecto>y4</numproyecto>
            <cantidad>700</cantidad>
        </suministra>
        ... omitido ...
    </suministros>
</datos>
\end{verbatim}


\end{document}
