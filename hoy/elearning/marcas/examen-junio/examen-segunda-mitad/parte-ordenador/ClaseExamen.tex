\documentclass{examen}
\usepackage{listings}
\begin{document}

\modulo{Lenguajes de marcas -- PARTE ORDENADOR}

\pregunta{Dado el archivo XML que se muestra a continuaci�n escribir las siguientes consultas:
\begin{itemize}
\end{itemize}

}{2}


\break
\pregunta{Dado el archivo XML que se puede encontrar a continuaci�n, 
crear una hoja de estilo XSLT que transforme dicho archivo en el archivo XML que aparece al final.
}{3}
\begin{verbatim}
<inventario><!--Archivo original-->
    <producto codigo="P1">
        <peso unidad="kg">10</peso>
        <nombre>Ordenador</nombre>
        <lugar edificio="B">
            <aula>10</aula>
        </lugar>
    </producto>
    <producto codigo="P2">
        <peso unidad='g'>500</peso>
        <nombre>Switch</nombre>
        <lugar edificio="A">
            <aula>6</aula>
        </lugar>
    </producto>
</inventario>
\end{verbatim}
\begin{verbatim}
<!--Esto debe ser lo que devuelva el archivo XSLT-->

<datos>
    <listacodigos>
        <codigo>P1</codigo>
        <codigo>P2</codigo>
    </listacodigos>
    <aulas>
        <aula>B-10</aula>
        <aula>A-6</aula>
    </aulas>
    <productos>
        <producto>
            Ordenador de 10kg
        </producto>
        <producto>
            Switch de peso 500g
        </producto>
    </productos>
</datos>

\end{verbatim}

\end{document}
