\documentclass{examen}
\usepackage{listings}
\begin{document}

\modulo{Lenguajes de marcas -- PARTE ORDENADOR (SOLO PARTE 2)}


\pregunta{Usando el mismo fichero XML con alumnos que se us� en la tarea de XQuery
y el software que se te ha proporcionado construye las consultas XQuery
que resuelven los siguientes problemas:
\begin{itemize}
\item{Devuelve los apellidos y nombre de los alumnos de manera ordenada ascendente y sin que aparezca la etiqueta {\tt apenom}}
\item{Devuelve el nombre de las asignaturas cuyo codigo sea {\tt a1}, {\tt a3} o {\tt a4}}
\item{Obtener el recuento de calificaciones aprobadas (es decir, la nota es mayor o igual que 5)}

\end{itemize}
Recuerda que {\bf no necesitas teclear el fichero XML. El programa JXMLTool lo puede
cargar usando el menu ``Ejemplos'' y dentro de �l ``Alumnos''}. En las consultas
el fichero se debe llamar siempre ``datos.xml'' por lo que en tus consultas tendr�s
que poner cosas como {\tt doc(``datos.xml'')/clase...}
}{1.5}




\break
\pregunta{Dado el archivo XML que se puede encontrar a continuaci�n, 
crear una hoja de estilo XSLT que transforme dicho archivo en el archivo XML que aparece al final.
}{3.5}
\begin{verbatim}
<inventario><!--Archivo original-->
    <producto codigo="P1">
        <peso unidad="kg">10</peso>
        <nombre>Ordenador</nombre>
        <lugar edificio="B">
            <aula>10</aula>
        </lugar>
    </producto>
    <producto codigo="P2">
        <peso unidad='g'>500</peso>
        <nombre>Switch</nombre>
        <lugar edificio="A">
            <aula>6</aula>
        </lugar>
    </producto>
</inventario>
\end{verbatim}
\begin{verbatim}
<!--Esto debe ser lo que devuelva el archivo XSLT-->

<datos>
    <listacodigos>
        <codigo>P1</codigo>
        <codigo>P2</codigo>
    </listacodigos>
    <aulas>
        <aula>B-10</aula>
        <aula>A-6</aula>
    </aulas>
    <productos>
        <producto>
            Ordenador de 10kg
        </producto>
        <producto>
            Switch de peso 500g
        </producto>
    </productos>
</datos>

\end{verbatim}

\end{document}
