\documentclass{examen}
\usepackage{listings}
\begin{document}

\modulo{Lenguajes de marcas -- PARTE ORDENADOR}

\pregunta{Crea una p�gina HTML que se modifique con CSS y que contenga los siguientes
elementos
\begin{itemize}
\item{Dos encabezamientos h1 y dos encabezamientos h2. Modifica con CSS y el atributo
{\tt class} su aspecto para que aparezcan escritos con un color distinto.}
\item{Crea una lista numerada con 3 elementos. Modifica el tipo de letra del primero y el tercero usando {\tt id}}
\item{Crea una tabla de 3 filas y 4 columnas. Dentro de cada celda puedes escribir cualquier cosa. Usando CSS y la t�cnica que prefieras modifica el fondo de las celdas de la columna que quieras}
\end{itemize}
}{2}
\break
\pregunta{Dado el archivo XML que se puede encontrar a continuaci�n, 
crear una hoja de estilo XSLT que genere el HTML necesario para que se extraigan en forma de lista ordenada los datos (plataforma, cantidad de RAM y tama�o) de los tablets cuya RAM sea menor de 3 GB.
}{3}
\begin{verbatim}
<pedido><!--FICHERO ORIGINAL-->
    <portatiles>
        <portatil>
            <peso>1430</peso>
            <ram unidad="MB">4096</ram>
            <disco tipo="ssd">500</disco>
            <precio>499</precio>
        </portatil>
        <portatil>
            <peso>1830</peso>
            <ram unidad="GB">6</ram>
            <disco tipo="ssd">1000</disco>
            <precio>1199</precio>
        </portatil>
        <portatil>
            <peso>1250</peso>
            <ram unidad="MB">2048</ram>
            <disco tipo="ssd">750</disco>
            <precio>699</precio>
        </portatil>
    </portatiles>
    <tablets>
        <tablet>
            <plataforma>Android</plataforma>
            <caracteristicas>
                <memoria medida="GB">2</memoria>
                <tamanio medida="pulgadas">6</tamanio>
                <bateria>LiPo</bateria>
            </caracteristicas>
        </tablet>
        <tablet>
            <plataforma>iOS</plataforma>
            <caracteristicas>
                <memoria medida="GB">4</memoria>
                <tamanio medida="pulgadas">9</tamanio>
                <bateria>LiIon</bateria>
            </caracteristicas>
        </tablet>
    </tablets>
</pedido>
\end{verbatim}
\begin{verbatim}
<!--Esto debe ser lo que devuelva el archivo XSLT-->
<html>
  <head>
    <title>Ejercicio 1</title>
  </head>
  <body>
    <h1>Resultado</h1>
    <table border="1">
      <tr>
        <td>Precio:499</td>
        <td>Memoria:4096</td>
        <td>Disco duro:500</td>
      </tr>
      <tr>
        <td>Precio:1199</td>
        <td>Memoria:6</td>
        <td>Disco duro:1000</td>
      </tr>
      <tr>
        <td>Precio:699</td>
        <td>Memoria:2048</td>
        <td>Disco duro:750</td>
      </tr>
      <tr>
        <td>iOS</td>
        <td>4</td>
        <td>LiIon</td>
      </tr>
    </table>
  </body>
</html>

\end{verbatim}

\end{document}
