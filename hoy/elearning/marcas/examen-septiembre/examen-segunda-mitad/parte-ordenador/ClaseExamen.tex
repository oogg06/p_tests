\documentclass{examen}
\usepackage{listings}
\begin{document}

\modulo{Lenguajes de marcas -- PARTE ORDENADOR (SOLO PARTE 2)}


\pregunta{Usando el mismo fichero XML con alumnos que se us� en la tarea de XQuery
y el software que se te ha proporcionado construye las consultas XQuery
que resuelven los siguientes problemas:
\begin{itemize}
\item{Extraer los nombres de todas las asignaturas ordenadas por orden alfab�tico ascendente.}
\item{Devuelve el nombre de las asignaturas cuyo codigo sea {\tt a1}, {\tt a3} o {\tt a4}}
\item{Hacer el recuento de personas que vivan en Miera (es decir, comprobar su elemento {\tt pobla})}

\end{itemize}
Recuerda que {\bf no necesitas teclear el fichero XML. El programa JXMLTool lo puede
cargar usando el menu ``Ejemplos'' y dentro de �l ``Alumnos''}. En las consultas
el fichero se debe llamar siempre ``datos.xml'' por lo que en tus consultas tendr�s
que poner cosas como {\tt doc(``datos.xml'')/clase...}
}{1.5}




\break
\pregunta{Dado el archivo XML que se puede encontrar a continuaci�n, 
crear una hoja de estilo XSLT que transforme dicho archivo en el archivo XML que aparece al final. Observa que en el resultado solo aparecer� un producto {\bf si est� en el edificio A}
}{3.5}
\begin{verbatim}
<inventario><!--Archivo original-->
    <producto codigo="P1">
        <peso unidad="kg">10</peso>
        <nombre>Ordenador</nombre>
        <lugar edificio="B">
            <aula>10</aula>
        </lugar>
    </producto>
    <producto codigo="P2">
        <peso unidad='g'>500</peso>
        <nombre>Switch</nombre>
        <lugar edificio="A">
            <aula>6</aula>
        </lugar>
    </producto>
</inventario>
\end{verbatim}
\begin{verbatim}
<!--Esto debe ser lo que devuelva el archivo XSLT-->
<resultado>
    <producto codigo="P2">
        <peso>500 g</peso>
        <nombre>Switch</nombre>
        <ubicacion>A6</ubicacion>
    </producto>
</resultado>

\end{verbatim}

\end{document}
