\documentclass{examen}
\usepackage{listings}
\begin{document}

\modulo{Lenguajes de marcas -- PARTE ESCRITA (SOLO PARTE 2)}


\pregunta{Usando el fichero XML con alumnos que se adjunta usar XQuery para crear consultas
que resuelven los siguientes problemas:
\begin{itemize}
\item{Extraer los nombres de todas las asignaturas ordenadas por orden alfab�tico ascendente.}
\item{Devuelve el nombre de las asignaturas cuyo codigo sea {\tt a1}, {\tt a3} o {\tt a4}}
\item{Hacer el recuento de personas que vivan en Miera (es decir, comprobar su elemento {\tt pobla})}

\end{itemize}

}{1.5}


\break
\begin{verbatim}
<!--Fichero para consultas XQuery-->
<modulo>
    <alumnos>
        <alumno cod="n12344345">
            <apenom>Alcalde Garc�a, Luis</apenom>
            <direc>Las Manos, 24</direc>
            <pobla>Lamadrid</pobla>
            <telef>942756645</telef>
        </alumno>
        <alumno cod="n43483437">
            <apenom>Gonz�lez P�rez, Olga</apenom>
            <direc>Miraflor 28 - 3A</direc>
            <pobla>Torres</pobla>
            <telef>942564355</telef>
        </alumno>
        <alumno cod="n88234942">
            <apenom>Fern�ndez D�az, Mar�a</apenom>
            <direc>Luisa Fernanda 53</direc>
            <pobla>Miera</pobla>
            <telef>942346945</telef>
        </alumno>   
    </alumnos>
    <asignaturas>
        <nombre cod="a1">FH</nombre>
        <nombre cod="a2">FOL</nombre>
        ...
    </asignaturas>
    <notas>
        <calificacion alum="n12344345" asig="a1">4</calificacion>
        <calificacion alum="n43483437" asig="a1">5</calificacion>
        ..Omitido..
    </notas>
</modulo>
\end{verbatim}

\break

\pregunta{Dado el archivo XML que se puede encontrar a continuaci�n, 
crear una hoja de estilo XSLT que transforme dicho archivo en el archivo XML que aparece al final. Observa que en el resultado solo aparecer� un producto {\bf si est� en el edificio A}
}{3.5}
\begin{verbatim}
<inventario><!--Archivo original para transformar con XSLT-->
    <producto codigo="P1">
        <peso unidad="kg">10</peso>
        <nombre>Ordenador</nombre>
        <lugar edificio="B">
            <aula>10</aula>
        </lugar>
    </producto>
    <producto codigo="P2">
        <peso unidad='g'>500</peso>
        <nombre>Switch</nombre>
        <lugar edificio="A">
            <aula>6</aula>
        </lugar>
    </producto>
</inventario>
\end{verbatim}
\begin{verbatim}
<!--Esto debe ser lo que devuelva el archivo XSLT-->
<resultado>
    <producto codigo="P2">
        <peso>500 g</peso>
        <nombre>Switch</nombre>
        <ubicacion>A6</ubicacion>
    </producto>
</resultado>

\end{verbatim}

\end{document}
