\documentclass[a4paper, 12pt]{examen}

\begin{document}

%\modulo{Prog. multim. y de dispositivos moviles}
%\modulo{Planif. y adm. de redes}
%\modulo{Prog. de servicios y procesos}
\modulo{Lenguajes de marcas}

%%%%%%%%%%%%%%%%%%%%%%%%%%%%%%%%%%%%%%%%%%%%%%%%%%%%%%%%%%%%
%%%%%%%%%%%% Recuerda guardar el fichero como ISO-8859-1
%%%%%%%%%%%%%%%%%%%%%%%%%%%%%%%%%%%%%%%%%%%%%%%%%%%%%%%%%%%%

\pregunta{ Elabora un fichero CSS que consiga exactamente
lo que se muestra en la figura \ref{figura1}. Ten en cuenta lo siguiente:

\begin{itemize}
\item{Debes usar el {\tt float} para posicionar las cajas.}
\item{Los elementos {\tt h1} de las cajas 1 y 2 van subrayados.}
\item{Puedes a�adir otros elementos HTML o atributos pero {\bf no puedes cambiar el orden}.}
\item{Todas las cajas llevan color de fondo, que en concreto son dos tonos. Un tono es para las
cajas 1 y 2 y el otro para la 3, 4 y la 5}
\item{Se ha modificado el espacio que hay entre el borde las cajas y el texto interno, en concreto se
han usado 20px de espacio.}
\end{itemize}

}{2}
\begin{figure}[h]
    \caption{Resultado esperable en el ejercicio 1}
    \label{figura1}
    \includegraphics[width=\linewidth]{ej1.png}
\end{figure}
\break


\pregunta{ Elabora un fichero CSS que consiga lo siguiente para el HTML mostrado.
\begin{itemize}
\item{El fondo de la p�gina siempre es un archivo de imagen llamado ``fondo.jpg''}
\item{Puedes a�adir otros elementos HTML o atributos pero {\bf no puedes cambiar el orden}.}
\item{Se han definido dos estilos: uno para pantallas grandes (de 800px en adelante) y otro para peque?as (de menos de 800px)}
\item{En pantallas peque�as:}
    \begin{itemize}
    \item{Se ha modificado el color de los encabezados y se ha usado un tono gris.}
    \item{Se ha cambiado la distancia entre cada caja con respecto a su parte de arriba en 15px.}
    \item{La caja 3 no se muestra (fondo blanco con texto blanco). Las cajas 2 y 4 tienen un aspecto distinto al de las cajas 1 y 5.}
    \end{itemize}
\item {En el caso de pantallas grandes:}

\begin{itemize}
    \item{El tipo de letra usado es Arial.}
    \item{{\bf Usando float} las cajas se alinean una al lado de la otra en el orden: 2, 1, 3, 5, 4.}
\end{itemize}
\end{itemize}

}{  5 }
\break
\begin{figure}[h]
    \caption{Resultado esperable en el ejercicio 2 (caso 1)}
    \label{figura31}
    \includegraphics[width=0.8\linewidth]{ej31.png}
\end{figure}

\begin{figure}[h]
    \caption{Resultado esperable en el ejercicio 2 (caso 2)}
    \label{figura32}
    \includegraphics[width=\linewidth]{ej32.png}
\end{figure}


\break
\begin{verbatim}
<div id="caja1"> <!--Ejercicio 1-->
    <h1>Caja 1</h1>
    Texto ...
</div><!--Fin de la caja 1-->
<div id="caja2">
    <h1>Caja 2</h1>
    Texto ...
</div><!--Fin de la caja 2-->
<div id="caja3">
    <h1>Caja 3</h1>
    Texto ...
</div><!--Fin de la caja 3-->
<div id="caja4">
    <h1>Caja 4</h1>
    Texto ...
</div><!--Fin de la caja 4-->
<div id="caja5">
    <h1>Caja 5</h1>
    Texto ...
</div><!--Fin de la caja 5-->
\end{verbatim}

\begin{verbatim}
<div id="caja1"> <!--Ejercicio 2-->
    <h1>Caja 1</h1>
    Texto ...
</div><!--Fin de la caja 1-->
<div id="caja2">
    <h1>Caja 2</h1>
    Texto ...
</div><!--Fin de la caja 2-->
<div id="caja3">
    <h1>Caja 3</h1>
    Texto ...
</div><!--Fin de la caja 3-->
<div id="caja4">
    <h1>Caja 4</h1>
    Texto ...
</div><!--Fin de la caja 4-->
<div id="caja5">
    <h1>Caja 5</h1>
    Texto ...
</div><!--Fin de la caja 5-->
\end{verbatim}


\pregunta{Dada la tabla HTML que se muestra (y que no se puede modificar) usar CSS para conseguir que se muestre como indica la figura:}{3}

\begin{itemize}
\item{El fondo de las cabeceras es azul y el color de las letras es blanco.}
\item{Consigue los bordes que se muestran en la figura.}
\item{La tabla se ha decorado variando las filas pares e impares.}
\item{Si se pasa el rat�n por las celdas de una fila, todas las celdas de dicha fila cambian de color para ayudar al usuario a ver en qu� fila est� situado.}
\end{itemize}
\break
\begin{verbatim}
        <table>
            <thead>
                <tr>
                    <th>Cabecera 1</th><th>Cabecera 2</th><th>Cabecera 3</th>
                </tr>
            </thead>
            <tbody>
                <tr>
                    <td>Dato</td><td>Dato</td><td>Dato</td>
                </tr>
                <tr>
                    <td>Dato</td><td>Dato</td><td>Dato</td>
                </tr>
                <tr>
                    <td>Dato</td><td>Dato</td><td>Dato</td>
                </tr>
                <tr>
                    <td>Dato</td><td>Dato</td><td>Dato</td>
                </tr>
            </tbody>
        </table>
\end{verbatim}
\begin{figure}[h]
\includegraphics[scale=0.3]{Tabla01.png}
\end{figure}

\end{document}
