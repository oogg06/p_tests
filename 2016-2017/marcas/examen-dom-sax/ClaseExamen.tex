\documentclass{examen}

\begin{document}
\modulo{Lenguajes de marcas}

\pregunta{Usando DOM procesar el fichero XML que se proporciona al final y averigue cuantos futuros hay cuya ciudad de procedencia sea Madrid y cuantos bonos hay cuyo precio S� sea estable. El programa deber� decirnos cual de las dos cantidades es mayor. Puedes utilizar la clase Java proporcionada y a�adirle los m�todos o c�digo que desees. No hace falta que copies el c�digo que ya se te proporciona.}{6.5}

\pregunta{Usando SAX procesar el mismo fichero y comprobar si hay alg�n bono cuya ciudad de procedencia sea ``Tokio'' y tengan un valor minimo de 9000 o m�s. Puedes utilizar la clase Java proporcionada y a�adirle los m�todos o c�digo que desees. No hace falta que copies el c�digo que ya se te proporciona.}{3.5}

\break 
\begin{verbatim}
public class ProcesadorSAX extends DefaultHandler{
  @Override
  public void startElement(
      String ns, String nombreCuandoHayNS, String nombreCuandoNoHayNS,
      Attributes atributos) throws SAXException
  {
    
  } //Fin de startElement
 
  @Override
  public void endElement(String ns, String nombreCuandoHayNS,
      String nombreCuandoNoHayNS) throws SAXException
  {
  
  } //Fin de endElement

  public static void main(String[] args) throws ParserConfigurationException,
      SAXException, IOException {
    SAXParserFactory fabrica;
    fabrica=SAXParserFactory.newInstance();
    SAXParser parser=fabrica.newSAXParser();
    XMLReader lector=parser.getXMLReader();
    lector.setContentHandler(new ProcesadorSAX());
    lector.parse("archivo.xml");
  } //Fin de main
} //Fin de la clase\end{verbatim}

\break

\begin{verbatim}
<listado>
	<futuro precio="11.28">
                <producto>Cafe</producto>
                <mercado>Am�rica Latina</mercado>
                <ciudad_procedencia>
                        <madrid/>
                </ciudad_procedencia>
        </futuro>
        <divisa precio="183">
                <nombre_divisa>Libra esterlina</nombre_divisa>
                <tipo_de_cambio>2.7:1 euros</tipo_de_cambio>
                <tipo_de_cambio>1:0.87 d�lares</tipo_de_cambio>
                <ciudad_procedencia>
                        <madrid/>
                </ciudad_procedencia>
        </divisa>

        <bono precio="10000" estable="si">
                <pais_de_procedencia>
                        Espa�a
                </pais_de_procedencia>
                <valor_deseado>9980</valor_deseado>
                <valor_minimo>9950</valor_minimo>
                <valor_maximo>10020</valor_maximo>
                <ciudad_procedencia>
                        <tokio/>
                </ciudad_procedencia>
        </bono>
        <letra precio="45020">
                <tipo_de_interes>4.54%</tipo_de_interes>
                <pais_emisor>
                        <espania/>
                </pais_emisor>
                <ciudad_procedencia>
                        <madrid/>
                </ciudad_procedencia>
        </letra>
</listado>
\end{verbatim}
\break
\begin{verbatim}
public class ProcesadorXML {
        public Node extraerRaiz(String nombreArchivo){
                DocumentBuilderFactory fabrica;
                DocumentBuilder constructor;
                Document documentoXML=null;
                File fichero=new File(nombreArchivo);
                fabrica=
                        DocumentBuilderFactory.newInstance();
                System.out.println("Procesando "+nombreArchivo);
                try {
                        constructor=
                          fabrica.newDocumentBuilder();
                        documentoXML=constructor.parse(fichero);
                } catch (Exception e) {
                        // TODO Auto-generated catch block
                        e.printStackTrace();
                }
                return documentoXML.getDocumentElement();
        } //Fin de extraerRaiz
} //Fin de ProcesadorXML
\end{verbatim}
\end{document}
