\documentclass{examen}

\begin{document}
\modulo{Lenguajes de marcas}

\pregunta{Elaborar una DTD que permita validar un fichero XML que se atenga a las reglas siguientes:}{4}
\begin{itemize}
\item{El elemento ra�z se llama {\tt ventas}. Dentro de �l debe haber 1 o m�s elementos {\tt venta}}
\item{Dentro de un elemento {\tt venta} debe haber uno de tres posibles elementos: {\tt cemento}, {\tt ladrillos} o {\tt azulejos}. Toda venta lleva siempre un atributo llamado {\tt c�digo}}
\item{Dentro de un elemento {\tt cemento} puede haber (o no) en primer lugar un elemento vac�o llamado {\tt portepagado} y despues debe haber un elemento llamado {\tt peso}. El peso puede tener (o no) un atributo llamado {\tt unidad}.}
\item{Dentro del elemento {\tt ladrillos} debe haber siempre un elemento llamado {\tt modelo} y puede haber o no un {\tt peso}, cuyas reglas son exactamente las mismas del {\tt peso} anterior.}
\item{Dentro de {\tt azulejos} debe haber en primer lugar un elemento {\tt color} y despues debe haber un elemento {\tt peso}. El {\tt peso} tiene las mismas reglas de antes. El elemento color puede llevar o no un atributo llamado {\tt escala}}
\item{A modo de ejemplo se muestra al final  un fichero XML de ejemplo que deber�a validarse correctamente con la DTD.}
\end{itemize}

\break

\pregunta{Crear un fichero de esquema XML que permita validar un fichero XML como el mostrado al final y para el cual se han definido las siguientes reglas}{6}
\begin{itemize}
\item{Debe haber un elemento ra�z llamado {\tt componentes} que debe contener uno o m�s elementos {\tt componente}}
\item{Un componente puede ser dos cosas: un elemento {\tt discoduro} o un elemento {\tt discossd}. Cualquiera de los dos debe llevar siempre un atributo {\tt capacidad} y un atributo llamado cantidad. La cantidad debe ser un n�mero entre 0 y 32768. La capacidad debe ser siempre una de estas 3 cadenas: GB, TB  o PB}
\item{Dentro de {\tt discoduro} ocurre lo siguiente: puede haber un primer elemento llamado {\tt fabricante} dentro del cual puede haber una cadena cualquiera. Despues hay un elemento obligatorio llamado {\tt precio} que puede ser una cadena cualquiera}
\item{Dentro de {\tt discossd} puede haber un primer elemento optativo llamado {\tt fabricante} que puede contener una cadena cualquiera. Despues debe haber un elemento {\tt fabricaci�n} con un atributo optativo {\tt pa�s} que puede contener una cadena cualquiera. Dentro de {\tt fabricaci�n} el contenido puede ser cualquier cadena.}

\end{itemize}
\break
\begin{verbatim}
<ventas>
    <venta codigo="VV-0001">
        <cemento>
            <portepagado/>
            <peso unidad="kg">1000</peso>
        </cemento>
    </venta>
    <venta codigo="VV-0002">
        <ladrillos>
            <modelo>L-72</modelo>
        </ladrillos>
    </venta>
    <venta codigo="VV-0003">
        <azulejos>
            <color escala="rgb">00ffdd</color>
            <peso>150</peso>
        </azulejos>
    </venta>
</ventas>
\end{verbatim}

\begin{verbatim}
<componentes>
    <componente>
        <discoduro capacidad="GB" cantidad="500">
            <fabricante>Western Digital</fabricante>
            <precio>150 dolares</precio>
        </discoduro>
    </componente>
    <componente>
        <discossd capacidad="GB" cantidad="250">
            <fabricante>Seagate</fabricante>
            <fabricacion pais="China">Proceso litogr�fico</fabricacion>
        </discossd>
    </componente>
</componentes>
\end{verbatim}
\end{document}
