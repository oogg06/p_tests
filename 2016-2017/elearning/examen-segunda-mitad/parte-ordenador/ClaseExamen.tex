\documentclass{examen}
\usepackage{listings}
\begin{document}

\modulo{Lenguajes de marcas -- PARTE ORDENADOR}

\pregunta{Unos programadores necesitan estructurar la informaci�n que intercambiar�n los ficheros de sus aplicaciones para lo cual han determinado los requisitos siguientes que quieren plasmar en una DTD:}{2}
\begin{itemize}
\item{    La ra�z se llama {\tt listafacturas}.}
\item{    Dentro de la lista debe haber uno o m�s elementos {\tt factura}.}
\item{    Las facturas tienen un atributo {\tt fecha} que es optativo.}
\item{    Toda factura tiene un {\tt emisor}, que es un elemento obligatorio y que debe tener un atributo {\tt cif} que es obligatorio. Dentro de {\tt emisor} debe haber un elemento {\tt nombre}, que es obligatorio y puede o no haber un elemento {\tt volumenventas}.}
\item{    Toda factura debe tener un elemento {\tt pagador}, el cual tiene exactamente la misma estructura que emisor.}
\item{    Toda factura tiene un elemento {\tt importe}.}
\item{    Puede verse m�s adelante un ejemplo de fichero XML.}
\end{itemize}

\break

\pregunta{Dado el archivo XML que se puede encontrar al final, 
crear una hoja de estilo XSLT que genere el HTML necesario para que se extraigan en forma de lista ordenada los datos (plataforma, cantidad de RAM y tama�o) de los tablets cuya RAM sea menor de 3 GB.
}{3}
\begin{verbatim}
<pedido><!--FICHERO ORIGINAL-->
    <portatiles>
        <portatil>
            <peso>1430</peso>
            <ram unidad="MB">4096</ram>
            <disco tipo="ssd">500</disco>
            <precio>499</precio>
        </portatil>
        <portatil>
            <peso>1830</peso>
            <ram unidad="GB">6</ram>
            <disco tipo="ssd">1000</disco>
            <precio>1199</precio>
        </portatil>
        <portatil>
            <peso>1250</peso>
            <ram unidad="MB">2048</ram>
            <disco tipo="ssd">750</disco>
            <precio>699</precio>
        </portatil>
    </portatiles>
    <tablets>
        <tablet>
            <plataforma>Android</plataforma>
            <caracteristicas>
                <memoria medida="GB">2</memoria>
                <tamanio medida="pulgadas">6</tamanio>
                <bateria>LiPo</bateria>
            </caracteristicas>
        </tablet>
        <tablet>
            <plataforma>iOS</plataforma>
            <caracteristicas>
                <memoria medida="GB">4</memoria>
                <tamanio medida="pulgadas">9</tamanio>
                <bateria>LiIon</bateria>
            </caracteristicas>
        </tablet>
    </tablets>
</pedido>
\end{verbatim}
\begin{verbatim}
<!--Esto debe ser lo que devuelva el archivo XSLT-->
<html>
  <head>
    <title>Ejercicio 1</title>
  </head>
  <body>
    <h1>Resultado</h1>
    <table border="1">
      <tr>
        <td>Precio:499</td>
        <td>Memoria:4096</td>
        <td>Disco duro:500</td>
      </tr>
      <tr>
        <td>Precio:1199</td>
        <td>Memoria:6</td>
        <td>Disco duro:1000</td>
      </tr>
      <tr>
        <td>Precio:699</td>
        <td>Memoria:2048</td>
        <td>Disco duro:750</td>
      </tr>
      <tr>
        <td>iOS</td>
        <td>4</td>
        <td>LiIon</td>
      </tr>
    </table>
  </body>
</html>

\end{verbatim}

\end{document}
