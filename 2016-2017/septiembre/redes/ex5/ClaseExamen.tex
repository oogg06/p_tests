\documentclass{examen}

\begin{document}
\modulo{Redes}

\pregunta{Explica como TCP puede usar las retransmisiones para evitar fallos}{2}
\pregunta{�Qu� son los puertos bien conocidos? �qu� valores toman?}{2}
\pregunta{Un administrador administra una red cuyas direcciones son del tipo 10.0.0.0/8. Dentro hay un servidor web con la IP 10.0.0.10 y se desea evitar que personas de redes externas se conecten a dicho servidor tanto al servidor ``normal'' como al servidor ``seguro''. El administrador tambi�n desea que la m�quina con la IP 10.0.0.2 no pueda consultar los DNS de la IP 8.8.8.8 (el puerto DNS es el 53 UDP).Explica detalladamente qu� pasos tendr� que dar en el cortafuegos.}{4}
\pregunta{Explica los pasos que da TCP para abrir una conexi�n establecida.}{1}
\pregunta{�Qu� quiere decir que la capa de transporte haga ``multiplexaci�n''?}{1}


\end{document}
