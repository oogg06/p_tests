\documentclass{examen}

\begin{document}
\modulo{Redes}

\pregunta{Un cliente est� intentando empezar con una conexi�n y env�a al 
servidor el paquete SYN, sin embargo justo despues de que el SYN llegue correctamente
al servidor el cable se corta �qu� pasar� en el servidor y en el cliente?}{2}

\pregunta{Un administrador administra una red cuyas direcciones son del tipo 192.168.1.0/24. Dentro de la red hay un servidor de correo en el puerto TCP 25 y se desea que usuarios externos que est�n en la red 80.0.0.0/16 puedan conectarse
a dicho servidor para descargar su correo. Adem�s de esto, se desea que nuestras m�quinas no puedan conectarse a ning�n servicio FTP (puertos TCP 20 y 21) que est�n en el exterior. Explica detalladamente las operaciones a realizar
en el cortafuegos.}{4}
\pregunta{Si tenemos un servidor web en el puerto TCP 80 virtualizado dentro de una m�quina virtual con VirtualBox �qu� tendremos que hacer para que una m�quina externa pueda conectarse a dicho puerto TCP 80?}{1}
\pregunta{�Qu� hace el comando netstat?}{1}
\pregunta{�En qu� se diferencian los puertos bien conocidos de los registrados?}{2}

\end{document}
