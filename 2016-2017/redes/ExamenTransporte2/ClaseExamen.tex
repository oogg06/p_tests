\documentclass{examen}

\begin{document}
\modulo{Redes}

\pregunta{Indica todas las categor�as de puertos indicando los posibles valores que puede tomar cada grupo}{2}

\pregunta{Un administrador administra una red cuyas direcciones son del tipo 62.41.128.0/20. Dentro de la red hay los sistemas siguientes:

\begin{itemize}
\item{Un servidor de bases de datos que opera en el puerto 3306 en la direcci�n 62.41.128.10}
\item{Un servidor con una aplicaci�n que trabaja en el puerto 5432 y en la direcci�n 62.41.128.11}
\item{Un conjunto de ordenadores que est� en la red 179.99.101.0/25}
\end{itemize}
En la parte exterior de la empresa est�n las siguientes redes de inte?es:
\begin{itemize}
\item{La red 161.192.224.0/26 que pertenece a otra sede de la empresa.}
\item{La red 59.21.0.0/12 que pertenece a unos clientes de la empresa y que tiene un servidor en la IP 59.21.0.1/12 y puerto 8195.}
\end{itemize}
Dada esta situaci�n la empresa desea que su cortafuegos haga cumplir los siguientes requisitos.
\begin{itemize}
\item{Se desea permitir que la otra sede de la empresa pueda conectarse a ambos servidores de esta sede}
\item{Se desea que los ordenadores de nuestra empresa puedan conectarse al servidor de  los clientes}
\item{Cuando llegue tr�fico de cualquier otra red deseamos que entre para poder registrarlo pero nunca se permitir� el tr�fico de respuesta}
\end{itemize}
}{5}
\pregunta{Si tenemos un servidor web en el puerto TCP 80 virtualizado dentro de una m�quina virtual con VirtualBox �qu� tendremos que hacer para que una m�quina externa pueda conectarse a dicho puerto TCP 80?}{1}
\pregunta{Explica las diferencias entre TCP y UDP}{1}
\pregunta{�Como funciona el cierre de conexiones en TCP?}{1}

\end{document}
