\documentclass{examen}

\begin{document}
\modulo{Redes}

\pregunta{Di todo lo que sepas sobre el NAT}{0.5}
\pregunta{Una empresa necesita crear 4 subredes llamadas A, B, C y D. La subred A tiene 43 equipos, la B tiene 88, la C tiene 39 y la D 126. El prefijo que les han asignado es 00000111.10000011.10001xxx.xxxxxxxx. Indica las direcciones de red, de difusi�n y la primera y la �ltima IP de cada subred en el caso de que se pueda conseguir lo que pide esta empresa}{1.5}
\pregunta{Explica de la manera m�s completa posible qu� es el NAT y como funciona.}{1}
\pregunta{Convierte esta IPv6 a formato resumido {\tt fe80:a000:0000:0000:a300:a200:0000:0000}}{0.5}
\pregunta{Di todo lo que sepas sobre estas direcciones:
\begin{itemize}
\item{225.1.2.141}
\item{172.23.95.0}
\item{161.84.43.15}
\end{itemize}
}{1.5}
\pregunta{Una empresa tiene varias sedes con routers conectados con enrutamiento est�tico como muestra la figura. Rellena las tablas adjuntas con las IP, m�scaras y las rutas (incluye las m�tricas junto a la direcci�n de siguiente salto).}{5}


\end{document}
