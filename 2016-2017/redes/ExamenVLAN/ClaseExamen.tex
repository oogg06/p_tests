\documentclass{examen}

\begin{document}
\modulo{Planificacion y administracion de redes.}
\pregunta{Observa la red de la figura. Escribe los comandos necesarios para crear VLANs y los subinterfaces asociados de acuerdo a los requisitos siguientes:
\begin{itemize}
	\item{PC0 y PC4 est�n en la VLAN 10. Esta VLAN tendr� la IP 10.10.0.0 con m�scara 255.255.0.0}
	\item{PC1, PC5 y PC6 est�n en la VLAN 20.Esta VLAN tendr� la IP 10.20.0.0 con m�scara 255.255.0.0}
	\item{PC2, PC3 y PC7 est�n en la VLAN 30. Esta VLAN tendr� la IP 10.30.0.0 con m�scara 255.255.0.0}
	\item{Los servidores SMTP y HTTP est�n en su propia VLAN, la 100.Esta VLAN tendr� la IP 10.100.0.0 con m�scara 255.255.0.0}
\end{itemize}

.}{6.5}

\pregunta{Asigna las direcciones IP, m�scaras y gateways que sean necesarios para que todo funcione. Crea tambi�n los comandos necesarios para conseguir lo siguiente:
\begin{itemize}
\item{Solo la red 192.168.10.0/24 puede acceder desde el exterior al servidor HTTP.}
\item{Las redes 10.14.0.0/16 y la 10.161.0.0/16 pueden acceder al SMTP.}
\item{La red 172.16.0.0/16 pueden hacer ping a los equipos de la VLAN 20.}
\end{itemize}
.}{3.5}
\begin{figure}
\includegraphics[scale=0.4]{examen-img/vlan.png}
\end{figure}
\end{document}
