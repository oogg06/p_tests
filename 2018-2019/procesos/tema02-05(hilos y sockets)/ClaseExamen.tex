\documentclass[a4paper, 12pt]{examen}
\usepackage{inputenc}
\begin{document}

%\modulo{Prog. multim. y de dispositivos moviles}
%\modulo{Planif. y adm. de redes}
\modulo{Prog. de servicios y procesos. Recuperaci�n}
%\modulo{Lenguajes de marcas}
\pregunta{Se desea crear una aplicaci�n cliente/servidor que se base en hilos para obtener el m�ximo rendimiento. En concreto se desea disponer de un servicio de agenda telef�nica que permita almacenar y borrar nombres asociados a n�meros de tel�fono as� como buscar en la agenda tanto por nombres como por n�meros de tel�fono. 

Para ello se especifica el siguiente comportamiento de clientes y servidor.}{10}
\begin{itemize}
\item{Existen cuatros tipos de mensajes: ALMACENAR, CONSULTANORMAL, CONSULTAINVERSA y BORRAR}
\item{Si un cliente quiere almacenar un n�mero en la agenda tiene que enviar tres cosas: La palabra ALMACENAR, un nombre de persona y un n�mero. Un servidor que reciba este mensaje almacenar� ese nombre y ese n�mero de tel�fono (puede hacerse en memoria, no se necesitan ficheros ni bases de datos). El servidor contestar� OK si se hizo el almacenamiento o ERROR si hubo alg�n error o excepci�n al intentar hacer el almacenamiento.}
\item{Si un cliente quiere hacer una consulta buscando por nombre de persona enviar� dos cosas: La palabra CONSULTANORMAL y un nombre de persona. El servidor contestar� con una sola l�nea con el n�mero de tel�fono o la palabra DESCONOCIDO para indicar que no se conoce tal nombre.}
\item{Si un cliente quiere hacer una b�squeda por n�mero de tel�fono tiene que enviar dos cosas: La palabra CONSULTAINVERSA y un n�mero de tel�fono. El servidor contestar� con una sola l�nea con el nombre de persona o la palabra DESCONOCIDO para indicar que no se conoce el n�mero.}
\item{Si un cliente quiere borrar un n�mero tiene que enviar dos cosas: la palabra BORRAR y un nombre de persona. El servidor contestar� OK si se borr� el nombre con el tel�fono o la palabra DESCONOCIDO si no se conoc�a ese nombre de persona. No existe la posibilidad de borrar buscando por n�mero de tel�fono, se puede ignorar ese caso.}
\end{itemize}
\end{document}
