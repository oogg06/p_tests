\documentclass[a4paper, 12pt]{examen}
\usepackage{inputenc}
\begin{document}

%\modulo{Prog. multim. y de dispositivos moviles}
%\modulo{Planif. y adm. de redes}
\modulo{Prog. de servicios y procesos}
%\modulo{Lenguajes de marcas}


\pregunta{Un cierto programa desea simular como responden los
programas multiproceso en condiciones aleatorias, para ello se ha dise�ado
un experimento para ejecutarse en Java que nos d� algunos datos
para extraer las conclusiones pertinentes. En concreto se desea lo siguiente:
\begin{itemize}
\item{Hay 3 procesos escritores que aceptan n�meros pero solo uno a la vez. Una vez
que reciben un n�mero lo imprimen en pantalla. Cada proceso lleva la cuenta
de cuantos n�meros le han pedido imprimir.}
\item{Hay 10  procesos extractores que se dedican a extraer n�meros. En concreto
sacan 30 n�meros al azar entre 1 y 50. Una vez extra�do un n�mero eligen un
proceso al azar, le env�an el n�mero y esperan un tiempo al azar entre 200 y 500 
milisegundos}
\end{itemize}
Escribir el programa Java que haga la simulaci�n y que indique cuantos n�meros
ha recibido cada proceso escritor. 
 }{10}
\end{document}
