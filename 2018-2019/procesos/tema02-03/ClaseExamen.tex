\documentclass[a4paper, 12pt]{examen}
\usepackage{inputenc}
\begin{document}

%\modulo{Prog. multim. y de dispositivos moviles}
%\modulo{Planif. y adm. de redes}
\modulo{Prog. de servicios y procesos}
%\modulo{Lenguajes de marcas}
\pregunta{Se desea simular el comportamiento de un parking en el que hay 100 plazas de aparcamiento distribuidas en 2 plantas de 50 plazas cada una. Cuando un coche llega pueden pasar dos cosas:
\begin{itemize}

\item {Que el parking no tenga plazas libres. En ese caso, el coche se marcha y no vuelve.}
\item {Que el parking s� tenga plazas. En ese caso, el coche intentar� buscar primero una plaza libre en la planta 0, y si no hay buscar� en la planta 1. Si tampoco hay no se vuelve a hacer ninguna comprobaci�n y se marcha. Sin embargo, si encuentra alguna plaza, el coche se queda un tiempo al azar en ella. Una vez pasado ese tiempo, el coche se marcha y no vuelve.}
\end{itemize}
}{10}

En cuanto a la simulaci�n se debe tener en cuenta lo siguiente:

\begin{itemize}


\item{Se permite el uso de funciones de biblioteca que ya se tengan en el ordenador (como por ejemplo la biblioteca Utilidades creada en clase que permite elegir n�meros al azar o esperar tiempos al azar) pero no se permite el uso de Internet.}

\item{Para acelerar las pruebas puede ser recomendable usar al principio menos procesos, y con menos tiempo de espera entre intentos. Para facilitar las cosas puede ser �til el usar constantes que nos permitan cambiar estas cantidades con facilidad.}

\item{Al finalizar se debe hacer un ZIP con las clases Java y enviarlas por correo al profesor.}
\end{itemize}
Escribir el programa Java que haga la simulaci�n de manera correcta.

\end{document}
