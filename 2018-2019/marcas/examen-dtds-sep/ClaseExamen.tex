\documentclass{examen}

\begin{document}
\modulo{Lenguajes de marcas. Tema 4. DTDs y XML Schemas}

\pregunta{Elaborar un XML Schema que permita validar un fichero XML como el mostrado al final y que se atenga a las reglas siguientes:}{6.5}
\begin{itemize}
\item{    El elemento ra�z es ``proveedores''}
\item{    Dentro de �l debe haber uno o m�s elementos ``proveedor''. Tiene siempre un atributo c�digo formado por 3 may�sculas, seguidas de un gui�n seguido de 3 cifras.}
\item{  Dentro de proveedor debe haber un elemento nombre dentro del cual habr� cadenas.}
\item{  Despues del proveedor hay siempre un elemento ubicacion y dentro de ubicaci�n hay siempre un atributo pa�s}
\item{  Despues hay un elemento credito, que es obligatorio. Dentro de �l puede haber o no un atributo divisa que solo acepta los valores ``euro'' o ``dolar''. Dentro de credito siempre hay un entero positivo.}
\item{  Despues puede haber o no un elemento llamado adicional. Si existe debe llevar siempre dos elementos: un elemento llamado ``web'' y despues un elemento ``telefono''}
\end{itemize}

\break

\pregunta{Crear una DTD que permita validar un fichero XML como el mostrado al final y para el cual se han definido las mismas reglas que para el ejercicio anterior.}{3.5}



\begin{verbatim}
<proveedores>
    <!--Debe haber uno o mas elementos proveedor. El atributo codigo es obligatorio-->
    <proveedor codigo="AAA-333">
        <!--Nombre obligatorio, sin hijos ni atributos-->
        <nombre>TESA</nombre>
        <!--El atributo pa�s es obligatorio-->
        <ubicacion pais="EEUU">Toledo</ubicacion>
        <!--Elemento credito optativo.Atributo divisa optativo, pero si existe
        solo se aceptan como divisa las cadenas euro, dolar o yen-->
        <credito divisa="euro">25000</credito>
        <!--Elemento adicional optativo-->
        <adicional>
            <!--Elemento web. Obligatorio-->
            <web>http://www.tesa.com</web>
            <!--Elemento telefono. Obligatorio-->
            <telefono>+34 555 22 11 33</telefono>
        </adicional>
    </proveedor>
</proveedores>
\end{verbatim}
\end{document}

