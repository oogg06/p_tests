\documentclass{examen}

\begin{document}

\modulo{Lenguajes de marcas -- PARTE ORDENADOR (SOLO PARTE 2)}


\pregunta{Usando el mismo fichero XML con alumnos que se us� en la tarea de XQuery y el software que se te ha proporcionado construye las consultas XQuery que resuelven los siguientes problemas:}{1.5}
\begin{itemize}
\item{Devuelve los apellidos y nombre de los alumnos de manera ordenada ascendente y sin que aparezca la etiqueta {\tt apenom}}
\item{Devuelve el nombre de las asignaturas cuyo codigo sea {\tt a1}, {\tt a3} o {\tt a4}}
\item{Obtener el recuento de notas aprobadas en la asignatura a1}

\end{itemize}

Recuerda que {\bf no necesitas teclear el fichero XML. El programa JXMLTool lo puede
cargar usando el menu ``Ejemplos'' y dentro de �l ``Alumnos''. En las consultas
el fichero se debe llamar siempre ``datos.xml'' por lo que en tus consultas tendr�s
que poner cosas como {\tt doc(``datos.xml'')/clase...} }





\break
\pregunta{Transformar con XSLT el archivo de pedido XML con el nombre ``Fichero original'' en el fichero que aparece m�s abajo. En la herramienta JXMLTool pueden encontrarse este archivo dentro del men� Ejemplos-Inventario. El programa debe buscar los productos que est�n en el edificio A y mostrarlos de la manera indicada abajo. En concreto los productos pasan a llamarse ``art�culos'' y dentro de ellos se ha creado un elemento ``ubicaci�n'' en el que aparece el nombre del edificio y el numero de aula.}{3.5}		

\begin{verbatim}
<!--FICHERO ORIGINAL-->
<inventario>
    <producto codigo="P1">
        <peso unidad="kg">10</peso>
        <nombre>Ordenador</nombre>
        <lugar edificio="B">
            <aula>10</aula>
        </lugar>
    </producto>
    <producto codigo="P2">
        <peso unidad='g'>500</peso>
        <nombre>Switch</nombre>
        <lugar edificio="A">
            <aula>6</aula>
        </lugar>
    </producto>
</inventario>

\end{verbatim}


\begin{verbatim}
<!--Fichero que debe salir como resultado del ejercicio XSLT-->
<listadoproductos>
    <articulo nombre="Ordenador">
        <ubicacion>A6</ubicacion>
    </articulo>
</listadoproductos>
\end{verbatim}
\end{document}
